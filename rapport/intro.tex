\newpage
\section{Proposal}

\subsection{Channel System}
A channel system CS over a set of processes P and channels Ch is a tuple $\langle$S,$s_0$,A,C,M,$\delta$$\rangle$, where
\begin{itemize}
\item[]
S is a finite set of control states,
\item[]
$s_0$ is an initial control state,
\item[]
A is a finite set of actions,
\item[]
Ch is a finite set of channels,
\item[]
M is a finite set of messages,
\item[]
$\delta$ is a finite set of transitions, each of which is a triple of the form $\langle s_1,op,s_2\rangle$, where $s_1$ and $s_2$ are control states, and op is a label of one of the forms

\begin{itemize}
\item
c!m, where c $\in$ Ch and m $\in$ M
\item
c?m, where c $\in$ Ch and m $\in$ M
\item
a $\in$ A.
\end{itemize}
\end{itemize}

The finite-state control part of CS is an ordinary labeled transition system with states S, initial state $s_0$ and transitions $\delta$. The channel part is represented by the set Ch of channels, which may contain a string of messages in M. The set A denotes the set of observable interactions with the environment, whereas $\delta$ may either perform an action from A, or and unobservable action, where

\begin{itemize}
\item[]
$\langle s_1, c!m, s_2\rangle$ represents a change of state from $s_1$ to $s_2$ while appending the message m to the tail of channel c
\item[]
$\langle s_1, c?m, s_2\rangle$ represents a change of state from $s_1$ to $s_2$ while removing the message m to the head of channel c
\end{itemize}

\subsection{Channel transition system}
The operational behaviour of CS is defined by the inifinite-state transition system TS = (C, $\rightarrow$) where
\begin{itemize}
\item[]
   K = (S $\times$ $\xi$) is the set of its configurations, where $\xi$ is an evaluation of the set of channels C in CS
\item[]
  $\rightarrow$ $\subseteq (S \times S)$ contains the following transitions
  \begin{itemize}
    \item
      For each observable action a $\in$ A in CS
      \[
      \dfrac{s \xrightarrow{a} s'}{(S, \xi) \rightarrow (S', \xi)}
      \]
    \item
      For each transmission action $\langle s_1, ch!m, s_2 \rangle$ in CS
      \[
      \dfrac{s \xrightarrow{ch!m} s' \wedge ch \in \xi}{(S, \xi) \rightarrow (S', \xi')} \] with \[ \xi' = \xi[ch := \xi (ch) \centerdot m].
      \]
    \item
      For each reception action $\langle s_1, ch?m, s_2 \rangle$ in CS
      \[
      \dfrac{s \xrightarrow{ch?m} s' \wedge ch \in \xi}{(S, m \centerdot \xi) \rightarrow (S', \xi')} \] with \[ \xi' = \xi[ch:= \xi (ch)].
      \]

  \end{itemize}
\end{itemize}

Let k denote a configuration, the k is said to be reachable in TS, if there are configurations $k_1...k_l$ such that $k_0$ is an initial conifuration of TS, $k_l = k$ and for each 0 $\leq$ i < l, $\langle k_i, k_{i+1} \rangle \in \rightarrow$.

\newpage

\subsection{Subwords and views}

Let $\subword$ be the subword relation, then u $\subword$ $s_1...s_n$=w iff u=$s_{i_1}...s_{i_k}$ where 1 $\leq$ $i_1,i_k$ $\leq $ n and $i_j < i_{j+1}$ for all 1 $leq$ j < k. For example, if w=abc, then the set of subwords of w is {abc, ab, bc, a, b, c}.

We define the subword relation $\subword$ for configurations, such that for c = $\confa{s}{\xi}$, c'=$\confa{s}{\xi[ ch \subword \xi(ch)] | ch \in \xi}$ .

\subsection{View abstraction}
The abstraction function $\alpha_p: C\rightarrow 2^{C_p}$!!!!!!!!! maps a configuration c into the set V of \emph{views} such that for each v $\in$ V, $\{v\sqsubseteq c\}$. 

The concretization function $\gamma_k: 2^{K_p} \rightarrow 2^K$ returns, set of views V, the set of configurations that can be reconstructed from the views in V, in other words, $\gamma_k(V) = \{c | \alpha_p(c) \subseteq V$\}

The abstract post-image of a set of view V $\in$ $C_p$ is defined as $Apost_p$(V) = $\alpha_p(post(\gamma_p(V)))$ We also define $\gamma_p^l$ := $\gamma_p(V) \cap C_l$.


\subsection{Example}
Consider the configuration k = ($s_1$, $r_1$, ab, ba) for the alternating bit protocol (defined elsewere), then $\alpha_2(k)$ = \{($s_1$, $r_1$, ab, ba), ($s_1$, $r_1$, ab, a), ($s_1$, $r_1$, ab, b), ($s_1$, $r_1$, b, ba), ($s_1$, $r_1$, a, ba), ($s_1$, $r_1$, ab, $\epsilon$), ($s_1$, $r_1$, $\epsilon$, ba)\} = V. Then $\gamma(V)$ = V.

The post of the set V contains
\begin{itemize}
\item
($s_1$, $r_1$, ab, ba) $\rightarrow$ ($s_2$, $r_1$, ab, ba), since ($s_1$$\rightarrow$$s_2$) $\in$ A.
\item
($s_1$, $r_1$, ab, a) $\rightarrow$ ($s_2$, $r_1$, ab, a), since ($s_1$$\rightarrow$$s_2$) $\in$ A.
\item
($s_1$, $r_1$, ab, a) $\rightarrow$ ($s_1$, $r_2$, ab, $\epsilon$), since $\langle$$r_1$,$c_2?m$,$r_2$$\rangle$ $\in$ $\delta$.
\item
and so on.
\end{itemize}
