%%%%%%%%%%%%%%%%%%%%%%%%%%%%%%%%%%%%%%%%%
% Arsclassica Article
% LaTeX Template
% Version 1.0 (21/4/14)
%
% This template has been downloaded from:
% http://www.LaTeXTemplates.com
%
% Original author:
% Lorenzo Pantieri (http://www.lorenzopantieri.net) with extensive modifications by:
% Vel (vel@latextemplates.com)
%
% License:
% CC BY-NC-SA 3.0 (http://creativecommons.org/licenses/by-nc-sa/3.0/)
%
%%%%%%%%%%%%%%%%%%%%%%%%%%%%%%%%%%%%%%%%%

%----------------------------------------------------------------------------------------
%	PACKAGES AND OTHER DOCUMENT CONFIGURATIONS
%----------------------------------------------------------------------------------------

\documentclass[
10pt, % Main document font size
a4paper, % Paper type, use 'letterpaper' for US Letter paper
oneside, % One page layout (no page indentation)
%twoside, % Two page layout (page indentation for binding and different headers)
headinclude,footinclude, % Extra spacing for the header and footer
BCOR5mm, % Binding correction
]{scrartcl}


\hyphenation{Fortran hy-phen-ation} % Specify custom hyphenation points in words with dashes where you would like hyphenation to occur, or alternatively, don't put any dashes in a word to stop hyphenation altogether

%----------------------------------------------------------------------------------------
%	TITLE AND AUTHOR(S)
%----------------------------------------------------------------------------------------

\title {Test}
\date{} % An optional date to appear under the author(s)

%----------------------------------------------------------------------------------------

\begin{document}

%----------------------------------------------------------------------------------------
%	HEADERS
%----------------------------------------------------------------------------------------


%----------------------------------------------------------------------------------------
%	TABLE OF CONTENTS & LISTS OF FIGURES AND TABLES
%----------------------------------------------------------------------------------------

\maketitle % Print the title/author/date block

\setcounter{tocdepth}{2} % Set the depth of the table of contents to show sections and subsections only

\tableofcontents % Print the table of contents

\listoffigures % Print the list of figures

\listoftables % Print the list of tables

%----------------------------------------------------------------------------------------
%	ABSTRACT
%----------------------------------------------------------------------------------------

\section*{Abstract} % This section will not appear in the table of contents due to the star (\section*)

%----------------------------------------------------------------------------------------
%	AUTHOR AFFILIATIONS
%----------------------------------------------------------------------------------------

\let\thefootnote\relax\footnotetext{* \textit{Department of Biology, University of Examples, London, United Kingdom}}

\let\thefootnote\relax\footnotetext{\textsuperscript{1} \textit{Department of Chemistry, University of Examples, London, United Kingdom}}

%----------------------------------------------------------------------------------------

\newpage % Start the article content on the second page, remove this if you have a longer abstract that goes onto the second page

%----------------------------------------------------------------------------------------
%	INTRODUCTION
%----------------------------------------------------------------------------------------

\section{Proposal}
\subsection{Channel system}
A channel system L is a tuple $\langle$S,$s_0$,A,C,M,$\delta$$\rangle$, where
\begin{itemize}
\item[]
S is a finite set of control states,
\item[]
$s_0$ is an initial control state,
\item[]
A is a finite set of actions,
\item[]
C is a finite set of channels,
\item[]
M is a finite set of messages,
\item[]
$\delta$ is a finite set of transitions, each of which is a triple of the form $\rangle$s1,op,s2$\langle$, where s1 and s2 are control states, and op is a label of one of the forms

\begin{itemize}
\item
c!m, where c $\in$ C and m $\in$ M,
\item
c?m, where c $\in$ C and m $\in$ M,
\end{itemize}
\end{itemize}

\subsection{Channel transition system}

A channel system induces a transition system TS = (K, $\rightarrow$) where K=$(SC)^*$ and $\rightarrow$ $\subseteq$ K$\times$K is the transition relation.

Using $s_k$[i] to denote the state of the ith process and $c_k$[i] to denote the state of the ith channel of a configuration k=($s_k$, $c_k$), the transition relation $\rightarrow$ contains relations k $\rightarrow$ k' with

\begin{itemize}
  \item
    $s_k[i] \longrightarrow s_k'[i]$ for some i, with $\longrightarrow \in A$, and the states of all processes j$\neq$i and all channels remain unchanged by the transition.
  \item
    $c_k[i] \longrightarrow c_k'[i]$ for some i, with $\longrightarrow \in \delta$, and all other channels j$\neq$i and the states of all processes remain unchanged by the transition.
\end{itemize}

\subsection{Subwords and views}

Using $c_k[i]$ be a channel in a transition system k containing the word w=$w_1...w_l$ of length l, then the subword relation $\sqsubseteq$ is a relation (as described elsewere). Then a view v of a configuration k=($s_k,c_k$) (denoted v $\sqsubseteq$ k, with the symbol overloaded) is a tuple ($s_k, c_k'$) such that $c_k'$ $\sqsubseteq$ $c_k$.

\subsection{View abstraction}
The abstraction function $\alpha_p: K\rightarrow 2^{K_p}$ maps a configuration k into the set $\alpha_k(k) = \{v\in C_k | v\sqsubseteq K\}$. 

The concretization function $\gamma_k: 2^{K_p} \rightarrow 2^K$ inputs a set of views V $\subseteq$ $K_p$, and returns the set of configurations that can be reconstructed from the views in V, in other words, $\gamma_k(V) = \{k \in K | \alpha_p(k) \subseteq V$\}

The abstract post-image of a set of view V $\in$ $C_p$ is defined as $Apost_p$(V) = $\alpha_p(post(\gamma_p(V)))$ We also define $\gamma_p^l$ := $\gamma_p(V) \cap K_l$.



\subsection{Example}
Consider the configuration k = ($s_1$, $r_1$, ab, ba) for the alternating bit protocol (defined elsewere), then $\alpha_2(k)$ = \{($s_1$, $r_1$, ab, ba), ($s_1$, $r_1$, ab, a), ($s_1$, $r_1$, ab, b), ($s_1$, $r_1$, b, ba), ($s_1$, $r_1$, a, ba), ($s_1$, $r_1$, ab, $\epsilon$), ($s_1$, $r_1$, $\epsilon$, ba)\} = V. Then $\gamma(V)$ = V.

The post of the set V contains
\begin{itemize}
\item
($s_1$, $r_1$, ab, ba) $\rightarrow$ ($s_2$, $r_1$, ab, ba), since ($s_1$$\rightarrow$$s_2$) $\in$ A.
\item
($s_1$, $r_1$, ab, a) $\rightarrow$ ($s_2$, $r_1$, ab, a), since ($s_1$$\rightarrow$$s_2$) $\in$ A.
\item
($s_1$, $r_1$, ab, a) $\rightarrow$ ($s_1$, $r_2$, ab, $\epsilon$), since $\langle$$r_1$,$c_2?m$,$r_2$$\rangle$ $\in$ $\delta$.
\item
and so on.
\end{itemize}






\end{document}