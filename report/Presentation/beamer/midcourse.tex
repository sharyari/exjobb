\documentclass{beamer}

\title[Crisis] % (optional, only for long titles)
{Algorithmic Verification of Channel Machines Using Small Models}
\author[J, Sharyari | \emph{sharyari@gmail.com}] {Jonathan Sharyari}
\usetheme[numbers]{Uppsala}
\subject{Computer Science}
\subtitle{Mid-Course Meeting}

\date[2014-09-09] % appears in the bottom of the sidebar
{September $9^{th}$, 2014}
\institute[Dept. of Information Technology] % appears in the footline
{
  Department of Information Technology\\
  Uppsala University
}

\usepackage{biblatex}
\addbibresource{references.bib}
%\setbeamertemplate{footline}[text line]{%
%  \parbox{\linewidth}{\vspace*{-8pt}Algorithmic Verification of Channel Machines\hfill- \insertpagenumber -\hfill\insertshortauthor}}
%\setbeamertemplate{navigation symbols}{}

\begin{document}
\begin{frame}[plain]
  \titlepage
\end{frame}
\section{Verification}
\begin{frame}
  \frametitle{Verification}
  \begin{itemize}
  \item
    Verification is the “process of evaluating software to determine whether the products of a given development phase satisfy the conditions imposed at the start of that phase”\cite{ordbok}
  \item
    Model checking is the task of automatically verifying the correctness of a program, with regard to its specification.
  \item
    This is generally done through an exhaustive graph search
  \end{itemize}
\end{frame}
\section{Channel Systems}
\begin{frame}
  \frametitle{Channel Systems}
  \begin{itemize}
  \item
    A channel system is a system that relies on channels for its operation, e.g. communication protocols
  \item
    If channels are unbounded, the model checking of such protocols corresponds to searching an infinite graph
  \end{itemize}
\end{frame}
\section{Small Models}
\begin{frame}
  \frametitle{Small Models}
  \begin{itemize}
  \item
    One technique of overcoming this problem is to use small models
  \item
    For some types of problems, a small problem instance may exhibit all the relevant behavior of a larger system
  \item
    Using small models, undecidable verification problems can be made decidable
  \end{itemize}
\end{frame}
\section{Problem Formulation}
\begin{frame}
  \frametitle{Problem Formulation}
  \begin{itemize}
  \item
    Combine the ideas of small models with that of a well-known verification technique -- abstract interpretation – to be applicable on channel systems
  \item
    Implement the verification algorithm
  \end{itemize}
\end{frame}
\section{Remaining Work}
\begin{frame}
  \frametitle{Remaining Work}
  \begin{itemize}
  \item
    Complete the implementation
  \item
    Carry out case studies
  \item
    Carry out comparisons towards similar verification tools
  \end{itemize}
\end{frame}

\section{References}
\begin{frame}
  \frametitle{References}
  \printbibliography
\end{frame}

\begin{frame}[plain]
  \begin{centering}
    \pgfimage[height=\textheight]{uppsala_logo}
    \par
  \end{centering}
\end{frame}


\end{document}
