\newpage
\section{Abstract Interpretation}
\label{model}
Formalized by Cousot and Cousot\cite{cousot1977}, abstract interpretation techniques are techniques to approximating programs. Using information about the control and data flow, i.e. the semantics of a system, the program can be verified without ever generating the full transition system.

\todo[inline]{1. Lemma1 was initially in this section, should I move it back? 2. should these simply be moved to the end of the chapter?}
The goal of this chapter is to show that abstract interpretation techniques can be used in order to create an overapproximation of views $V$. This is done by defining a concretization function $\gamma$ and an abstraction function $\alpha$.

The approximation is said to be a valid abstraction, if, for any transition (function) over the set $V$ Lemma \ref{lemma1} holds. In this section, I first define the concept of a \emph{subword} needed to define the two functions $\alpha$ and $\gamma$. It is then proven that using these functions, Lemma \ref{lemma1} holds for all transitions, as stated in \ref{CTS}.


\subsection{Subwords and Views}
\label{subwords}
We define the \e{views} of a configurations \e{c} = $\conf{s,\xi}$, to be the set $V$ = \{$\conf{s, \xi'}$ such that $\xi' \subword \xi$.

We define size($ch$) to be equivalent to size($\xi(ch)$), i.e. the length of the word on the channel \e{ch}. We define the size of a configuration as well as the size of a view to equal the length of the longest word on its channels.

\paragraph{Example.} Suppose \e{c} is a configuration $\conf{s_1,s_2,ab,cd}$. The configuration is of size 2 and its views are

\begin{ttabular}
$\conf{s_1,s_2,ab,cd}$ \\
$\conf{s_1,s_2,a,cd}$ &
$\conf{s_1,s_2,b,cd}$ &
$\conf{s_1,s_2,\epsilon,cd}$ \\ 
$\conf{s_1,s_2,a,c}$ &
$\conf{s_1,s_2,b,c}$ &
$\conf{s_1,s_2,\epsilon,c}$ \\
$\conf{s_1,s_2,a,d}$ &
$\conf{s_1,s_2,b,d}$ &
$\conf{s_1,s_2,\epsilon,d}$ \\
$\conf{s_1,s_2,\epsilon,\epsilon}$ \\
\end{ttabular}


\subsection{Abstractions and Concretizations}
\label{alphagamma}
For a given parameter $k \in \mathbb{N}$, we use $C$ and $V$ to denote sets of configurations and views respectively, and $C_k$ and $V_k$ to denote configurations and views of size up to $k$.

The abstraction function $\alpha_k: C\rightarrow 2^{C_k}$\todo{?} maps a configuration \e{c} into the set \e{V} of views of size up to $k$, such that for each $v\in V$, $\{\xi(v)\sqsubseteq \xi(c)\}$ and $size(\xi(c)) \leq k$ . 

The concretization function $\gamma_k: 2^{C_k} \rightarrow 2^C$\todo{?} returns, given a set of views \e{V}, the set of configurations that can be reconstructed from the views in \e{V}, in other words, $\gamma_c(V) = \{c \in C$ | $\alpha_k(c) \subseteq V$\}

For a set \e{V}, we define the \e{post-image} of \e{V}, \e{post(V)} = \{$c'$ | \e{c} $\rightarrow$ \e{c'} $\wedge$ \e{c} $\in$ \e{V}\}. The \e{abstract post-image} of a set \e{V} $\subseteq$ $C_k$ is defined as $Apost_k$(\e{V}) = $\alpha_k(post(\gamma_k(V)))$. In general $\gamma_k(V)$ is an infinite set of configurations. We define $\gamma_k^l(V)$ := $\gamma_k(V) \cap C_l$ for some $l\geq 0$. The intuitive meaning of $\gamma_k^l(V)$ is the set of $l$-size configurations for which all views of length at most $k$ are in $V$.

\subsection{Small Model Property}
\label{proof}
Calculating the abstract post-image of a set of views $V \subseteq C_k$ is essentiatial for the verification procedure. As $\gamma_k(V)$ typically is infinite, this cannot be done straightforwardly. It is the main result of \ref{parosh} that it suffices to consider configurations of $\gamma_k(V)$ of sizes up to $k+1$, which is a finite set of configurations for which the abstract post-image can be computed. Formally, they show that

\begin{lemma}
\label{lemma1}
For any $k\in\mathbb{N}$, and $V\subseteq C_k$, $\alpha_k(post(\gamma_k(V)))$ $\cup$ $V$ = $\alpha_k(post(\gamma_k^{k+1}(V)))$ $\cup$ $V$.
\end{lemma}

\todo{Why is this correct? Shouldn't it be the fixpoint of the two that is equal?}

Below, we show that this lemma holds in the context of lossy channel systems. We will show that for any configuration \e{c} $\in$ $\gamma_k(V)$ of size $m > k + 1$ such that there is a configuration \e{c'} where $c' \xrightarrow{r} c$, then for each \e{v'} $\in$ $\alpha_k(c')$, the following holds: There is a configuration \e{d} $\in$ $\gamma_k(V)$ of size at most \e{k}+1 with a transition \e{d} $\xrightarrow{r}$ \e{d'} with \e{v'} $\in$ $\alpha_k(d')$. 

\subsubsection{Proofs}

\paragraph{Transmission rules}
\label{proofTransmission}
First we note, that for any configuration \e{c} $\in$ \e{V}, any view \e{v'} $\in$ $\alpha_k(c)$ is also a valid configuration \e{v'} $\in$ $\gamma_k(V)$, since $\alpha_k(v')$ $\subseteq$ $\alpha_k(c)$ and thus \e{v'} $\in$ $\gamma_k^{k+1}(\alpha_k(c))$. Also note that if from \e{c} a transition \e{r} can be fired, then this transition can also be fired from any configuration \e{c'} = \e{v'}, as transmission rules are guarded only by the states of the channel system and not by channel evaluations.

A transmission rule changes the evaluation of at most one channel \e{ch} $\in$ \e{c}, and (possibly) the state of the channel system, thus we need only reason about the evaluations of a single channel \e{ch}.

Let \e{c} = $\conf{S, w}$ $\xrightarrow{ch!w_{m+1}}$ $\conf{S', w \bullet m}$ = \e{c'}.

The views of \e{c'} of size up to k are either of the type 1) $\conf{S', w' \sqsubset w}$, with size($w'$) $\leq$ $k$ (i.e. not including the newly transmitted message) or 2) of the form $\conf{S', w' \bullet m | w' \sqsubseteq w}$ with size($w'$) < $k$.

For any view of type 1, there exists a configuration of size $k$, \e{d} = $\conf{S, w'}$ $\in$ $\alpha_k{c}$ and the transition $r$ can be taken, resulting in \e{d'} = $\conf{S, w'\bullet m}$ of size $k$+1. The view \e{v'} $\in$ $\alpha_k{w'}$.

For any view of type 2, there exists a configuration of size $k-1$, \e{d} = $\conf{S, w'}$ and the transition $r$ can be taken resulting in \e{d'} = $\conf{S, w'\bullet m}$ = $v'$.

\e{Example}. Assume a system with two processes and a single channel. Let \e{c} = $\conf{1,2,abc}$ $\rightarrow{ch!d}$ $\conf{2,2,abcd}$. Assume that $\e{c}$ $\in$ $\gamma_2(V)$, then $\alpha_2(c)$ $\in$ \e{V}, i.e. 
$\conf{1,2,a}$, $\conf{1,2,b}$, $\conf{1,2,c}$, $\conf{1,2,ab}$, $\conf{1,2,bc}$, $\conf{1,2,ac}$ are in in V and also in $\gamma_k(V)$.

$\alpha_2(c')$ = \{$\conf{2,2,a}$, $\conf{2,2,b}$, $\conf{2,2,c}$, $\conf{2,2,d}$, $\conf{2,2,ab}$, $\conf{2,2,bc}$, $\conf{2,2,cd}$, $\conf{2,2,ac}$, $\conf{2,2,ad}$, $\conf{2,2,bd} $ \}. Consider a view with the newly transmitted message, $\conf{2,2,cd}$, it can by created by $\conf{1,2,c}$ $\rightarrow{ch!d}$ $\conf{2,2,cd}$. Considering instead a view without the transmitted message, $\conf{1,2,ab}$, it can be created by $\conf{1,2,ab}$ $\rightarrow$ $\conf{2,2,abd}$ for which $\conf{2,2,ab}$ is a view.

\paragraph{Reception rules}
\label{proofreception}
As opposed to the transmission rules, reception rules rely both on the state and the evaluation in order to be fired, but only the state of a single channel need be considered.

Consider a configration \e{c} = $\conf{S, m\bullet w}$ $\xrightarrow{ch?m}$ $\conf{S, w}$ = c', such that $size(c)$ $\geq$ $k+1$. For any view \e{v'} $\in$ $\alpha_k(c)$ with the word \e{w'} of size at most \e{k} on the channel, there exists a configuration $d$ of size at most \e{k+1}, \e{d} = $\conf{S, m\bullet w'}$ $\in$ $\alpha_k(c)$ such that \e{d} $\xrightarrow{ch?m}$ \e{d'} = \e{v'}.


\paragraph{Actions}
Actions can in this context be seen as equivalent to a reception rule, reading the empty symbol $\epsilon$ on some channel. The proof then follows directly from \ref{proofreception}.
