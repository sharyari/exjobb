\section{Summary and Further Work}
This paper has investigated the possibility of verifying systems depending on lossy channels. The approach was to model such systems as parameterized systems and to use abstract interpretation techniques in order to bound infinite systems, to a carefully chosen finite subset of states. This method is influenced by and builds upond previous work\cite{parosh}, adapting the existing verification method to work for unbounded lossy channels.

The verification method was implemented, and a protocol specification language was designed, influenced by the \textsc{xml}-layout used by the \textsc{mpass} verification language\cite{mpass}. Experimental results with the verifier, in comparison to that of the \textsc{mpass} verifier and a verification algorithm based on backward reachability analysis\cite{287591}, show that the method achieved comparable or better results on most of the protocols tested.

A notable weakness of the verification method is that it struggles with the time and space complexity caused by an increased number of messages, but copes relatively well with an increase of local states in the processes, as all reachable channel evaluations are explicitly stored. This is a point that could be improved, by finding a method to abstractly represent a larger number of channel evaluations more compactly, possibly by introducing further overapproximation.

In addition to this, an important theoretical issue has been largely overlooked in this paper, i.e. the fact that the verifier is not guaranteed to ever converge towards a solution. Considering that the verifier terminates for all tested protocols, it is likely that the verification algorithm could leave such a guarantee for at least some classes of problems, as is done in \cite{parosh}. Identifying and formally proving such a guarantee of termination would be a valuable theoretical addition worth investigation.
